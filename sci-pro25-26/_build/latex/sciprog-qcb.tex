%% Generated by Sphinx.
\def\sphinxdocclass{report}
\documentclass[letterpaper,10pt,english]{sphinxmanual}
\ifdefined\pdfpxdimen
   \let\sphinxpxdimen\pdfpxdimen\else\newdimen\sphinxpxdimen
\fi \sphinxpxdimen=.75bp\relax
\ifdefined\pdfimageresolution
    \pdfimageresolution= \numexpr \dimexpr1in\relax/\sphinxpxdimen\relax
\fi
%% let collapsible pdf bookmarks panel have high depth per default
\PassOptionsToPackage{bookmarksdepth=5}{hyperref}
%% turn off hyperref patch of \index as sphinx.xdy xindy module takes care of
%% suitable \hyperpage mark-up, working around hyperref-xindy incompatibility
\PassOptionsToPackage{hyperindex=false}{hyperref}
%% memoir class requires extra handling
\makeatletter\@ifclassloaded{memoir}
{\ifdefined\memhyperindexfalse\memhyperindexfalse\fi}{}\makeatother

\PassOptionsToPackage{booktabs}{sphinx}
\PassOptionsToPackage{colorrows}{sphinx}

\PassOptionsToPackage{warn}{textcomp}

\catcode`^^^^00a0\active\protected\def^^^^00a0{\leavevmode\nobreak\ }
\usepackage{cmap}
\usepackage{fontspec}
\defaultfontfeatures[\rmfamily,\sffamily,\ttfamily]{}
\usepackage{amsmath,amssymb,amstext}
\usepackage{polyglossia}
\setmainlanguage{english}



\setmainfont{FreeSerif}[
  Extension      = .otf,
  UprightFont    = *,
  ItalicFont     = *Italic,
  BoldFont       = *Bold,
  BoldItalicFont = *BoldItalic
]
\setsansfont{FreeSans}[
  Extension      = .otf,
  UprightFont    = *,
  ItalicFont     = *Oblique,
  BoldFont       = *Bold,
  BoldItalicFont = *BoldOblique,
]
\setmonofont{FreeMono}[Scale=0.9,
  Extension      = .otf,
  UprightFont    = *,
  ItalicFont     = *Oblique,
  BoldFont       = *Bold,
  BoldItalicFont = *BoldOblique,
]



\usepackage[Bjarne]{fncychap}
\usepackage{sphinx}

\fvset{fontsize=auto}
\usepackage{geometry}

\usepackage{nbsphinx}

% Include hyperref last.
\usepackage{hyperref}
% Fix anchor placement for figures with captions.
\usepackage{hypcap}% it must be loaded after hyperref.
% Set up styles of URL: it should be placed after hyperref.
\urlstyle{same}


\usepackage{sphinxmessages}



\usepackage{newunicodechar}
\usepackage{amsmath}
\usepackage{wasysym}
\usepackage{graphicx}

\makeatletter
% Actually APLlog is a a thin star against white circle, could't find anything better
\newunicodechar{✪}{\APLlog}    
\newunicodechar{✓}{\checkmark}    

    

\title{DS Scientific Programming Lab}
\date{Sep 20, 2025}
\release{dev}
\author{Antonio Longa}
\newcommand{\sphinxlogo}{\vbox{}}
\renewcommand{\releasename}{Release}
\makeindex
\begin{document}

\pagestyle{empty}

            \makeatletter
            \pagestyle{empty}
            \thispagestyle{empty}
            \noindent\rule{\textwidth}{1pt}\par
                \begingroup % for PDF information dictionary
                \def\endgraf{ }\def\and{\& }%
                \pdfstringdefDisableCommands{\def\\{, }}% overwrite hyperref setup
                \hypersetup{pdfauthor={\@author}, pdftitle={\@title}}%
                \endgroup
            \begin{flushright}
                \sphinxlogo
                \py@HeaderFamily
                {\Huge \@title }\par
            {\itshape\large The webpage of the Scientific Programming Lab for DS}\par
                \vspace{25pt}
                {\Large
                \begin{tabular}[t]{c}
                    \@author
                \end{tabular}}\par
                \vspace{25pt}
                \@date \par
                \py@authoraddress \par
            \end{flushright}
            \@thanks
            \setcounter{footnote}{0}
            \let\thanks\relax\let\maketitle\relax
            %\gdef\@thanks{}\gdef\@author{}\gdef\@title{}
                \vfill
                \noindent Copyright \copyright\ \the\year\ by \@author.
                \vskip 10pt
                \noindent \@title\ is available under the Creative Commons Attribution 4.0
                International License, granting you the right to copy, redistribute, modify, and
                sell it, so long as you attribute the original to \@author\ and identify any
                changes that you have made. Full terms of the license are available at:
                \vskip 10pt
                \noindent \url{http://creativecommons.org/licenses/by/4.0/}
                \vskip 10pt
                \noindent The complete book can be found online for free at:
                \vskip 10pt
                \noindent \url{https://jupman.softpython.org/en/latest/}
\pagestyle{plain}
\sphinxtableofcontents
\pagestyle{normal}
\phantomsection\label{\detokenize{toc-page::doc}}
\sphinxstepscope

\sphinxstepscope




\chapter{General Info}
\label{\detokenize{index:General-Info}}
\sphinxAtStartPar
For any additional information you can write an email:
\begin{itemize}
\item {} 
\sphinxAtStartPar
for technical support and for help with the exercises to the tutors:
\begin{itemize}
\item {} 
\sphinxAtStartPar
\sphinxstylestrong{Erich: erich dot robbi at unitn dot it}

\end{itemize}

\end{itemize}


\section{Timetable and lecture rooms}
\label{\detokenize{index:Timetable-and-lecture-rooms}}
\sphinxAtStartPar
The teaching will be divided in \sphinxstylestrong{two modules} according to the theoretical part provided by Prof. Passerini (1st module) and Prof. Dassi (2nd module). Each module will be composed of \sphinxstylestrong{12 lessons (6 weeks)} that also comprehend \sphinxstylestrong{2 mid\sphinxhyphen{}term exams}, one between the two modules (beginning of november) and one at the end (mid december).


\section{Equipment}
\label{\detokenize{index:Equipment}}
\sphinxAtStartPar
Even if the lessons are provided in a computer science lab, \sphinxstylestrong{students are highly encouraged to bring their personal laptop to work independently in solving the exercises}.


\section{Acknowledgements}
\label{\detokenize{index:Acknowledgements}}
\sphinxAtStartPar
I would also like to thank \sphinxstylestrong{Luca Marchetti, David Leoni, Stefano Teso, Luca Bianco, and Erik Dassi} for allowing me to use their material to prepare the course.

\sphinxstepscope


\chapter{Module 1, Practical 1}
\label{\detokenize{M1_practical1:Module-1,-Practical-1}}\label{\detokenize{M1_practical1::doc}}
\sphinxAtStartPar
The practicals of the first teaching module are a refinement of those prepared by Luca Bianco. Many thanks for his help.

\sphinxAtStartPar
The aim of this practical is to set up a working Python3.x development environment and will start familiarizing a bit with Python.


\section{Setting up the environment}
\label{\detokenize{M1_practical1:Setting-up-the-environment}}
\sphinxAtStartPar
We will need to install several pieces of software to get a working programming environment suitable for this practical. In this section we will install everything that we are going to need in the next few weeks.

\sphinxAtStartPar
Python3 is available for Windows, Linux and Mac, therefore you can run it on your preferred platform.


\subsection{Our toolbox}
\label{\detokenize{M1_practical1:Our-toolbox}}
\sphinxAtStartPar
If you decide to work on Windows or Mac, you can safely skip the following information and go straight to the section \sphinxstylestrong{“Installing Python3 in Windows/Mac”}. Note that, regardless your operating system, a useful source of information on how to install python can be found \sphinxhref{http://docs.python-guide.org/en/latest/}{here}%
\begin{footnote}[1]\sphinxAtStartFootnote
\sphinxnolinkurl{http://docs.python-guide.org/en/latest/}
%
\end{footnote}.


\subsection{Installing Visual studio Code (VScode) in Linux}
\label{\detokenize{M1_practical1:Installing-Visual-studio-Code-(VScode)-in-Linux}}
\sphinxAtStartPar
Please follow the following \sphinxhref{https://youtu.be/NX8SHmkuLn4}{video tutorial VScode Ubuntu}%
\begin{footnote}[2]\sphinxAtStartFootnote
\sphinxnolinkurl{https://youtu.be/NX8SHmkuLn4}
%
\end{footnote}.


\subsection{Installing Visual studio Code (VScode) in Windows}
\label{\detokenize{M1_practical1:Installing-Visual-studio-Code-(VScode)-in-Windows}}
\begin{DUlineblock}{0em}
\item[] Please follow the following \sphinxhref{https://www.youtube.com/watch?v=mIVB-SNycKI}{video tutorial VScode Windows}%
\begin{footnote}[3]\sphinxAtStartFootnote
\sphinxnolinkurl{https://www.youtube.com/watch?v=mIVB-SNycKI}
%
\end{footnote}. Follow the tutorial up to \sphinxstylestrong{minute 5:27}.
\item[] The rest is interesting but \sphinxstylestrong{not required} for our course.
\end{DUlineblock}


\subsection{Installing Visual studio Code (VScode) in MAC}
\label{\detokenize{M1_practical1:Installing-Visual-studio-Code-(VScode)-in-MAC}}
\sphinxAtStartPar
Please follow the following \sphinxhref{https://www.youtube.com/watch?v=NirAuEAblvo}{video tutorial VScode Mac}%
\begin{footnote}[4]\sphinxAtStartFootnote
\sphinxnolinkurl{https://www.youtube.com/watch?v=NirAuEAblvo}
%
\end{footnote}.


\section{The console}
\label{\detokenize{M1_practical1:The-console}}
\sphinxAtStartPar
To access the console on Linux just open a terminal and type:

\sphinxAtStartPar
\sphinxcode{\sphinxupquote{python3}}

\sphinxAtStartPar
while in Windows you have to look for “Python” and run “Python 3.x”. The console should look like this:

\noindent\sphinxincludegraphics[width=591\sphinxpxdimen,height=360\sphinxpxdimen]{{pythonconsole}.png}

\sphinxAtStartPar
Now we are all set to start interacting with the Python interpreter. In the console, type the following instructions (i.e. the first line and then press ENTER):

\begin{sphinxuseclass}{nbinput}
{
\begin{sphinxVerbatim}[commandchars=\\\{\}]
\llap{\color{nbsphinxin}[2]:\,\hspace{\fboxrule}\hspace{\fboxsep}}\PYG{l+m+mi}{5} \PYG{o}{+} \PYG{l+m+mi}{3}
\end{sphinxVerbatim}
}

\end{sphinxuseclass}
\begin{sphinxuseclass}{nboutput}
\begin{sphinxuseclass}{nblast}
{

\kern-\sphinxverbatimsmallskipamount\kern-\baselineskip
\kern+\FrameHeightAdjust\kern-\fboxrule
\vspace{\nbsphinxcodecellspacing}

\sphinxsetup{VerbatimColor={named}{white}}
\begin{sphinxuseclass}{output_area}
\begin{sphinxuseclass}{}


\begin{sphinxVerbatim}[commandchars=\\\{\}]
\llap{\color{nbsphinxout}[2]:\,\hspace{\fboxrule}\hspace{\fboxsep}}8
\end{sphinxVerbatim}



\end{sphinxuseclass}
\end{sphinxuseclass}
}

\end{sphinxuseclass}
\end{sphinxuseclass}
\sphinxAtStartPar
All as expected. The “In {[}1{]}” line is the input, while the “Out {[}1{]}” reports the output of the interpreter. Let’s challenge python with some other operations:

\begin{sphinxuseclass}{nbinput}
{
\begin{sphinxVerbatim}[commandchars=\\\{\}]
\llap{\color{nbsphinxin}[3]:\,\hspace{\fboxrule}\hspace{\fboxsep}}\PYG{l+m+mi}{12} \PYG{o}{/} \PYG{l+m+mi}{5}
\end{sphinxVerbatim}
}

\end{sphinxuseclass}
\begin{sphinxuseclass}{nboutput}
\begin{sphinxuseclass}{nblast}
{

\kern-\sphinxverbatimsmallskipamount\kern-\baselineskip
\kern+\FrameHeightAdjust\kern-\fboxrule
\vspace{\nbsphinxcodecellspacing}

\sphinxsetup{VerbatimColor={named}{white}}
\begin{sphinxuseclass}{output_area}
\begin{sphinxuseclass}{}


\begin{sphinxVerbatim}[commandchars=\\\{\}]
\llap{\color{nbsphinxout}[3]:\,\hspace{\fboxrule}\hspace{\fboxsep}}2.4
\end{sphinxVerbatim}



\end{sphinxuseclass}
\end{sphinxuseclass}
}

\end{sphinxuseclass}
\end{sphinxuseclass}
\begin{sphinxuseclass}{nbinput}
{
\begin{sphinxVerbatim}[commandchars=\\\{\}]
\llap{\color{nbsphinxin}[4]:\,\hspace{\fboxrule}\hspace{\fboxsep}}\PYG{l+m+mi}{1}\PYG{o}{/}\PYG{l+m+mi}{133}
\end{sphinxVerbatim}
}

\end{sphinxuseclass}
\begin{sphinxuseclass}{nboutput}
\begin{sphinxuseclass}{nblast}
{

\kern-\sphinxverbatimsmallskipamount\kern-\baselineskip
\kern+\FrameHeightAdjust\kern-\fboxrule
\vspace{\nbsphinxcodecellspacing}

\sphinxsetup{VerbatimColor={named}{white}}
\begin{sphinxuseclass}{output_area}
\begin{sphinxuseclass}{}


\begin{sphinxVerbatim}[commandchars=\\\{\}]
\llap{\color{nbsphinxout}[4]:\,\hspace{\fboxrule}\hspace{\fboxsep}}0.007518796992481203
\end{sphinxVerbatim}



\end{sphinxuseclass}
\end{sphinxuseclass}
}

\end{sphinxuseclass}
\end{sphinxuseclass}
\begin{sphinxuseclass}{nbinput}
{
\begin{sphinxVerbatim}[commandchars=\\\{\}]
\llap{\color{nbsphinxin}[5]:\,\hspace{\fboxrule}\hspace{\fboxsep}}\PYG{l+m+mi}{2}\PYG{o}{*}\PYG{o}{*}\PYG{l+m+mi}{1000}
\end{sphinxVerbatim}
}

\end{sphinxuseclass}
\begin{sphinxuseclass}{nboutput}
\begin{sphinxuseclass}{nblast}
{

\kern-\sphinxverbatimsmallskipamount\kern-\baselineskip
\kern+\FrameHeightAdjust\kern-\fboxrule
\vspace{\nbsphinxcodecellspacing}

\sphinxsetup{VerbatimColor={named}{white}}
\begin{sphinxuseclass}{output_area}
\begin{sphinxuseclass}{}


\begin{sphinxVerbatim}[commandchars=\\\{\}]
\llap{\color{nbsphinxout}[5]:\,\hspace{\fboxrule}\hspace{\fboxsep}}10715086071862673209484250490600018105614048117055336074437503883703510511249361224931983788156958581275946729175531468251871452856923140435984577574698574803934567774824230985421074605062371141877954182153046474983581941267398767559165543946077062914571196477686542167660429831652624386837205668069376
\end{sphinxVerbatim}



\end{sphinxuseclass}
\end{sphinxuseclass}
}

\end{sphinxuseclass}
\end{sphinxuseclass}
\sphinxAtStartPar
And some assignments:

\begin{sphinxuseclass}{nbinput}
{
\begin{sphinxVerbatim}[commandchars=\\\{\}]
\llap{\color{nbsphinxin}[6]:\,\hspace{\fboxrule}\hspace{\fboxsep}}\PYG{n}{a} \PYG{o}{=} \PYG{l+m+mi}{10}
\PYG{n}{b} \PYG{o}{=} \PYG{l+m+mi}{7}
\PYG{n}{s} \PYG{o}{=} \PYG{n}{a} \PYG{o}{+} \PYG{n}{b}
\PYG{n}{d} \PYG{o}{=} \PYG{n}{a} \PYG{o}{/} \PYG{n}{b}

\PYG{n+nb}{print}\PYG{p}{(}\PYG{l+s+s2}{\PYGZdq{}}\PYG{l+s+s2}{sum is:}\PYG{l+s+s2}{\PYGZdq{}}\PYG{p}{,}\PYG{n}{s}\PYG{p}{,} \PYG{l+s+s2}{\PYGZdq{}}\PYG{l+s+s2}{ division is:}\PYG{l+s+s2}{\PYGZdq{}}\PYG{p}{,}\PYG{n}{d}\PYG{p}{)}
\end{sphinxVerbatim}
}

\end{sphinxuseclass}
\begin{sphinxuseclass}{nboutput}
\begin{sphinxuseclass}{nblast}
{

\kern-\sphinxverbatimsmallskipamount\kern-\baselineskip
\kern+\FrameHeightAdjust\kern-\fboxrule
\vspace{\nbsphinxcodecellspacing}

\sphinxsetup{VerbatimColor={named}{white}}
\begin{sphinxuseclass}{output_area}
\begin{sphinxuseclass}{}


\begin{sphinxVerbatim}[commandchars=\\\{\}]
sum is: 17  division is: 1.4285714285714286
\end{sphinxVerbatim}



\end{sphinxuseclass}
\end{sphinxuseclass}
}

\end{sphinxuseclass}
\end{sphinxuseclass}
\sphinxAtStartPar
In the first four lines, values have been assigned to variables through the = operator. In the last line, the print function is used to display the output. For the time being, we will skip all the details and just notice that the print function somehow managed to get text and variables in input and coherently merged them in an output text. Although quite useful in some occasions, the console is quite limited therefore you can close it for now. To exit press Ctrl\sphinxhyphen{}D or type exit() and press ENTER.


\section{VScode}
\label{\detokenize{M1_practical1:VScode}}
\sphinxAtStartPar
Open VScode and create a file name \sphinxcode{\sphinxupquote{test.py}}, copy and paste the following code and try to execute it!

\begin{sphinxuseclass}{nbinput}
{
\begin{sphinxVerbatim}[commandchars=\\\{\}]
\llap{\color{nbsphinxin}[2]:\,\hspace{\fboxrule}\hspace{\fboxsep}}\PYG{l+s+sd}{\PYGZdq{}\PYGZdq{}\PYGZdq{}}
\PYG{l+s+sd}{This is the first example of Python script.}
\PYG{l+s+sd}{\PYGZdq{}\PYGZdq{}\PYGZdq{}}
\PYG{n}{a} \PYG{o}{=} \PYG{l+m+mi}{10} \PYG{c+c1}{\PYGZsh{} variable a}
\PYG{n}{b} \PYG{o}{=} \PYG{l+m+mi}{33} \PYG{c+c1}{\PYGZsh{} variable b}
\PYG{n}{c} \PYG{o}{=} \PYG{n}{a} \PYG{o}{/} \PYG{n}{b} \PYG{c+c1}{\PYGZsh{} variable c holds the ratio}

\PYG{c+c1}{\PYGZsh{} Let\PYGZsq{}s print the result to screen.}
\PYG{n+nb}{print}\PYG{p}{(}\PYG{l+s+s2}{\PYGZdq{}}\PYG{l+s+s2}{a:}\PYG{l+s+s2}{\PYGZdq{}}\PYG{p}{,} \PYG{n}{a}\PYG{p}{,} \PYG{l+s+s2}{\PYGZdq{}}\PYG{l+s+s2}{ b:}\PYG{l+s+s2}{\PYGZdq{}}\PYG{p}{,} \PYG{n}{b}\PYG{p}{,} \PYG{l+s+s2}{\PYGZdq{}}\PYG{l+s+s2}{ a/b=}\PYG{l+s+s2}{\PYGZdq{}}\PYG{p}{,} \PYG{n}{c}\PYG{p}{)}
\end{sphinxVerbatim}
}

\end{sphinxuseclass}
\begin{sphinxuseclass}{nboutput}
\begin{sphinxuseclass}{nblast}
{

\kern-\sphinxverbatimsmallskipamount\kern-\baselineskip
\kern+\FrameHeightAdjust\kern-\fboxrule
\vspace{\nbsphinxcodecellspacing}

\sphinxsetup{VerbatimColor={named}{white}}
\begin{sphinxuseclass}{output_area}
\begin{sphinxuseclass}{}


\begin{sphinxVerbatim}[commandchars=\\\{\}]
a: 10  b: 33  a/b= 0.30303030303030304
\end{sphinxVerbatim}



\end{sphinxuseclass}
\end{sphinxuseclass}
}

\end{sphinxuseclass}
\end{sphinxuseclass}
\sphinxAtStartPar
A couple of things worth nothing: the first three lines opened and closed by “”” are some text describing the content of the script. Moreover, comments are proceeded by the hash key (\#) and they are just ignored by the python interpreter.

\begin{sphinxadmonition}{note}{Note}\par
\sphinxAtStartPar
Good \sphinxstyleemphasis{Pythonic} code follows some syntactic rules on how to write things, naming conventions etc. The IDE will help you writing pythonic code even though we will not enforce this too much in this course. If you are interested in getting more details on this, you can have a look at the \sphinxhref{https://www.python.org/dev/peps/pep-0008/}{PEP8 Python Style Guide}%
\begin{footnote}[5]\sphinxAtStartFootnote
\sphinxnolinkurl{https://www.python.org/dev/peps/pep-0008/}
%
\end{footnote} (Python Enanchement Proposals \sphinxhyphen{} index 8).
\end{sphinxadmonition}

\begin{sphinxadmonition}{warning}{Warning}\par
\sphinxAtStartPar
\sphinxstylestrong{Please remember to comment your code, as it helps readability and will make your life easier when you have to modify or just understand the code you wrote some time in the past.}
\end{sphinxadmonition}

\sphinxAtStartPar
Please notice that Visual Studio Code will help you writing your Python scripts. For example, when you start writing the \sphinxstylestrong{print} line it will complete the code for you (\sphinxstylestrong{if the Pylint extension mentioned above is installed}), suggesting the functions that match the letters typed in. This useful feature is called \sphinxstylestrong{code completion} and, alongside suggesting possible matches, it also visualizes a description of the function and parameters it needs. Here is an example:

\noindent\sphinxincludegraphics[width=696\sphinxpxdimen,height=247\sphinxpxdimen]{{codecompletion}.png}

\sphinxAtStartPar
Save the file (Ctrl+S as shortcut). It is convenient to ask the IDE to highlight potential \sphinxstyleemphasis{syntactic} problems found in the code. You can toggle this function on/off by clicking on \sphinxstylestrong{View –> Problems}. The \sphinxstyleemphasis{Problems} panel should look like this

\noindent\sphinxincludegraphics[width=959\sphinxpxdimen,height=594\sphinxpxdimen]{{problems}.png}

\sphinxAtStartPar
Visual Studio Code is warning us that the variable names \sphinxstyleemphasis{a,b,c} at lines 4,5,6 do not follow Python naming conventions for constants (do you understand why? Check \sphinxhref{https://www.python.org/dev/peps/pep-0008/\#constants}{here}%
\begin{footnote}[6]\sphinxAtStartFootnote
\sphinxnolinkurl{https://www.python.org/dev/peps/pep-0008/\#constants}
%
\end{footnote} to find the answer). This warning is because they have been defined at the top level (there is no structure to our script yet) and therefore are interpreted as constants. The naming convention for constants states that they should be in capital letters. To amend the code,
you can just replace all the names with the corresponding capitalized name (i.e. A,B,C). If you do that, and you save the file again (Ctrl+S), you will see all these problems disappearing as well as the green underlining of the variable names. If your code does not have an empty line before the end, you might get another warning “\sphinxstyleemphasis{Final new line missing}”.

\begin{sphinxadmonition}{note}{Info}\par
\sphinxAtStartPar
Note that these were just warnings and the interpreter \sphinxstylestrong{in this case} will happily and correctly execute the code anyway, but it is always good practice to understand what the warnings are telling us before deciding to ignore them! Please, note also that these warnings could be not dispalyed with some versions of python and Visual Studio Code.
\end{sphinxadmonition}

\sphinxAtStartPar
Had we by mistake mispelled the \sphinxstylestrong{print} function name (something that should not happen with the code completion tool that suggests functions names!) writing \sphinxstyleemphasis{printt} (note the double t), upon saving the file, the IDE would have underlined in red the function name and flagged it up as a problem.

\noindent\sphinxincludegraphics[width=433\sphinxpxdimen,height=355\sphinxpxdimen]{{errors}.png}

\sphinxAtStartPar
This is because the builtin function \sphinxstyleemphasis{printt} does not exist and the python interpreter does not know what to do when it reads it. Note that \sphinxstyleemphasis{printt} is actually underlined in red, meaning that there is an error which will cause the interpreter to stop the execution with a failure. \sphinxstylestrong{Please remember ALL ERRORS MUST BE FIXED before running any piece of code.}

\sphinxAtStartPar
Now it is time to execute the code. By \sphinxstylestrong{right\sphinxhyphen{}clicking} in the code panel and selecting \sphinxstylestrong{Run Python File in Terminal} (see picture below) you can execute the code you have just written.

\noindent\sphinxincludegraphics[width=874\sphinxpxdimen,height=454\sphinxpxdimen]{{pythonrun}.png}

\sphinxAtStartPar
Upon clicking on \sphinxstyleemphasis{Run Python File in Terminal} a terminal panel should pop up in the lower section of the coding panel and the result shown above should be reported.

\sphinxAtStartPar
Saving script files like the \sphinxstylestrong{example1.py} above is also handy because they can be invoked several times (later on we will learn how to get inputs from the command line to make them more useful…). To do so, you just need to call the python intepreter passing the script file as parameter. From the folder containing the \sphinxstyleemphasis{example1.py} script:

\sphinxAtStartPar
\sphinxcode{\sphinxupquote{python3 example1.py}}

\sphinxAtStartPar
will in fact return:

\sphinxAtStartPar
a: 10 b: 33 a/b= 0.30303030303030304

\begin{sphinxadmonition}{note}{Info: syntactic vs semantic errors}\par
\sphinxAtStartPar
Before ending this section, let me add another note on errors. The IDE will diligently point you out \sphinxstylestrong{syntactic} warnings and errors (i.e. errors/warnings concerning the structure of the written code like name of functions, number and type of parameters, etc.) but it will not detect \sphinxstylestrong{semantic} or \sphinxstylestrong{runtime} errors (i.e. connected to the meaning of your code or to the value of your variables). These sort of errors will most probably make your code crash or may result in unexpected
results/behaviours. In the next section we will introduce the debugger, which is a useful tool to help detecting these errors.
\end{sphinxadmonition}

\sphinxAtStartPar
Before getting into that, consider the following lines of code (do not focus on the \sphinxstyleemphasis{import} line, this is only to load the mathematics module and use its method \sphinxstyleemphasis{sqrt} to compute the square root of its parameter):

\begin{sphinxuseclass}{nbinput}
{
\begin{sphinxVerbatim}[commandchars=\\\{\}]
\llap{\color{nbsphinxin}[7]:\,\hspace{\fboxrule}\hspace{\fboxsep}}\PYG{l+s+sd}{\PYGZdq{}\PYGZdq{}\PYGZdq{}}
\PYG{l+s+sd}{Runtime error example, compute square root of numbers}
\PYG{l+s+sd}{\PYGZdq{}\PYGZdq{}\PYGZdq{}}
\PYG{k+kn}{import} \PYG{n+nn}{math}

\PYG{n}{A} \PYG{o}{=} \PYG{l+m+mi}{16}
\PYG{n}{B} \PYG{o}{=} \PYG{n}{math}\PYG{o}{.}\PYG{n}{sqrt}\PYG{p}{(}\PYG{n}{A}\PYG{p}{)}
\PYG{n}{C} \PYG{o}{=} \PYG{l+m+mi}{5}\PYG{o}{*}\PYG{n}{B}
\PYG{n+nb}{print}\PYG{p}{(}\PYG{l+s+s2}{\PYGZdq{}}\PYG{l+s+s2}{A:}\PYG{l+s+s2}{\PYGZdq{}}\PYG{p}{,} \PYG{n}{A}\PYG{p}{,} \PYG{l+s+s2}{\PYGZdq{}}\PYG{l+s+s2}{ B:}\PYG{l+s+s2}{\PYGZdq{}}\PYG{p}{,} \PYG{n}{B}\PYG{p}{,} \PYG{l+s+s2}{\PYGZdq{}}\PYG{l+s+s2}{ C:}\PYG{l+s+s2}{\PYGZdq{}}\PYG{p}{,} \PYG{n}{C}\PYG{p}{)}

\PYG{n}{D} \PYG{o}{=} \PYG{n}{math}\PYG{o}{.}\PYG{n}{sqrt}\PYG{p}{(}\PYG{n}{A}\PYG{o}{\PYGZhy{}}\PYG{n}{C}\PYG{p}{)} \PYG{c+c1}{\PYGZsh{} whoops, A\PYGZhy{}C is now \PYGZhy{}4!!!}
\PYG{n+nb}{print}\PYG{p}{(}\PYG{n}{D}\PYG{p}{)}
\end{sphinxVerbatim}
}

\end{sphinxuseclass}
\begin{sphinxuseclass}{nboutput}
{

\kern-\sphinxverbatimsmallskipamount\kern-\baselineskip
\kern+\FrameHeightAdjust\kern-\fboxrule
\vspace{\nbsphinxcodecellspacing}

\sphinxsetup{VerbatimColor={named}{white}}
\begin{sphinxuseclass}{output_area}
\begin{sphinxuseclass}{}


\begin{sphinxVerbatim}[commandchars=\\\{\}]
A: 16  B: 4.0  C: 20.0
\end{sphinxVerbatim}



\end{sphinxuseclass}
\end{sphinxuseclass}
}

\end{sphinxuseclass}
\begin{sphinxuseclass}{nboutput}
\begin{sphinxuseclass}{nblast}
{

\kern-\sphinxverbatimsmallskipamount\kern-\baselineskip
\kern+\FrameHeightAdjust\kern-\fboxrule
\vspace{\nbsphinxcodecellspacing}

\sphinxsetup{VerbatimColor={named}{white}}
\begin{sphinxuseclass}{output_area}
\begin{sphinxuseclass}{}


\begin{sphinxVerbatim}[commandchars=\\\{\}]
\textcolor{ansi-red}{---------------------------------------------------------------------------}
\textcolor{ansi-red}{ValueError}                                Traceback (most recent call last)
Input \textcolor{ansi-green}{In [7]}, in \textcolor{ansi-cyan}{<cell line: 11>}\textcolor{ansi-blue}{()}
\textcolor{ansi-green-intense}{\textbf{      8}} C \def\tcRGB{\textcolor[RGB]}\expandafter\tcRGB\expandafter{\detokenize{98,98,98}}{=} \def\tcRGB{\textcolor[RGB]}\expandafter\tcRGB\expandafter{\detokenize{98,98,98}}{5}\def\tcRGB{\textcolor[RGB]}\expandafter\tcRGB\expandafter{\detokenize{98,98,98}}{*}B
\textcolor{ansi-green-intense}{\textbf{      9}} \def\tcRGB{\textcolor[RGB]}\expandafter\tcRGB\expandafter{\detokenize{0,135,0}}{print}(\def\tcRGB{\textcolor[RGB]}\expandafter\tcRGB\expandafter{\detokenize{175,0,0}}{"}\def\tcRGB{\textcolor[RGB]}\expandafter\tcRGB\expandafter{\detokenize{175,0,0}}{A:}\def\tcRGB{\textcolor[RGB]}\expandafter\tcRGB\expandafter{\detokenize{175,0,0}}{"}, A, \def\tcRGB{\textcolor[RGB]}\expandafter\tcRGB\expandafter{\detokenize{175,0,0}}{"}\def\tcRGB{\textcolor[RGB]}\expandafter\tcRGB\expandafter{\detokenize{175,0,0}}{ B:}\def\tcRGB{\textcolor[RGB]}\expandafter\tcRGB\expandafter{\detokenize{175,0,0}}{"}, B, \def\tcRGB{\textcolor[RGB]}\expandafter\tcRGB\expandafter{\detokenize{175,0,0}}{"}\def\tcRGB{\textcolor[RGB]}\expandafter\tcRGB\expandafter{\detokenize{175,0,0}}{ C:}\def\tcRGB{\textcolor[RGB]}\expandafter\tcRGB\expandafter{\detokenize{175,0,0}}{"}, C)
\textcolor{ansi-green}{---> 11} D \def\tcRGB{\textcolor[RGB]}\expandafter\tcRGB\expandafter{\detokenize{98,98,98}}{=} \setlength{\fboxsep}{0pt}\colorbox{ansi-yellow}{math\strut}\def\tcRGB{\textcolor[RGB]}\expandafter\tcRGB\expandafter{\detokenize{98,98,98}}{\setlength{\fboxsep}{0pt}\colorbox{ansi-yellow}{.\strut}}\setlength{\fboxsep}{0pt}\colorbox{ansi-yellow}{sqrt\strut}\setlength{\fboxsep}{0pt}\colorbox{ansi-yellow}{(\strut}\setlength{\fboxsep}{0pt}\colorbox{ansi-yellow}{A\strut}\def\tcRGB{\textcolor[RGB]}\expandafter\tcRGB\expandafter{\detokenize{98,98,98}}{\setlength{\fboxsep}{0pt}\colorbox{ansi-yellow}{-\strut}}\setlength{\fboxsep}{0pt}\colorbox{ansi-yellow}{C\strut}\setlength{\fboxsep}{0pt}\colorbox{ansi-yellow}{)\strut} \def\tcRGB{\textcolor[RGB]}\expandafter\tcRGB\expandafter{\detokenize{95,135,135}}{\# whoops, A-C is now -4!!!}
\textcolor{ansi-green-intense}{\textbf{     12}} \def\tcRGB{\textcolor[RGB]}\expandafter\tcRGB\expandafter{\detokenize{0,135,0}}{print}(D)

\textcolor{ansi-red}{ValueError}: math domain error
\end{sphinxVerbatim}



\end{sphinxuseclass}
\end{sphinxuseclass}
}

\end{sphinxuseclass}
\end{sphinxuseclass}
\sphinxAtStartPar
If you add that code to a python file (e.g. sqrt\_example.py), you save it and you try to execute it, you should get an error message as reported above. You can see that the interpreter has happily printed off the vaule of A,B and C but then stumbled into an error at line 9 (math domain error) when trying to compute \(\sqrt{A-C} = \sqrt{-4}\), because the sqrt method of the math module cannot be applied to negative values (i.e. it works in the domain of real numbers).

\sphinxAtStartPar
\sphinxstyleemphasis{Please take some time to familiarize with Visual Studio Code (creating files, saving files etc.) as in the next practicals we will take this ability for granted.}


\section{The debugger}
\label{\detokenize{M1_practical1:The-debugger}}
\sphinxAtStartPar
Another important feature of advanced Integrated Development Environments (IDEs) is their debugging capabilities. Visual Studio Code comes with a debugging tool that can help you trace the execution of your code and understand where possible errors hide.

\begin{sphinxadmonition}{note}{Note}\par
\sphinxAtStartPar
Please note that the following part of the guide presents a set of commands that \sphinxstylestrong{could be different} from the ones of the most updated versions of Visual Studio Code. For example, in my current version of the editor the debugger can be launched with \sphinxstylestrong{Run –> Start Debugging} (shortcut F5). Before doing this, you should add a breakpoint to the code with \sphinxstylestrong{Run –> Toggle Breakpoint} (shortcut F9).
\end{sphinxadmonition}

\sphinxAtStartPar
Write the following code on a new file (let’s call it \sphinxstyleemphasis{integer\_sum.py}) and execute it to get the result.

\begin{sphinxuseclass}{nbinput}
{
\begin{sphinxVerbatim}[commandchars=\\\{\}]
\llap{\color{nbsphinxin}[8]:\,\hspace{\fboxrule}\hspace{\fboxsep}}\PYG{l+s+sd}{\PYGZdq{}\PYGZdq{}\PYGZdq{} integer\PYGZus{}sum.py is a script to}
\PYG{l+s+sd}{ compute the sum of the first 1200 integers. \PYGZdq{}\PYGZdq{}\PYGZdq{}}

\PYG{n}{S} \PYG{o}{=} \PYG{l+m+mi}{0}
\PYG{k}{for} \PYG{n}{i} \PYG{o+ow}{in} \PYG{n+nb}{range}\PYG{p}{(}\PYG{l+m+mi}{0}\PYG{p}{,} \PYG{l+m+mi}{1201}\PYG{p}{)}\PYG{p}{:}
    \PYG{n}{S} \PYG{o}{=} \PYG{n}{S} \PYG{o}{+} \PYG{n}{i}

\PYG{n+nb}{print}\PYG{p}{(}\PYG{l+s+s2}{\PYGZdq{}}\PYG{l+s+s2}{The sum of the first 1200 integers is: }\PYG{l+s+s2}{\PYGZdq{}}\PYG{p}{,} \PYG{n}{S}\PYG{p}{)}
\end{sphinxVerbatim}
}

\end{sphinxuseclass}
\begin{sphinxuseclass}{nboutput}
\begin{sphinxuseclass}{nblast}
{

\kern-\sphinxverbatimsmallskipamount\kern-\baselineskip
\kern+\FrameHeightAdjust\kern-\fboxrule
\vspace{\nbsphinxcodecellspacing}

\sphinxsetup{VerbatimColor={named}{white}}
\begin{sphinxuseclass}{output_area}
\begin{sphinxuseclass}{}


\begin{sphinxVerbatim}[commandchars=\\\{\}]
The sum of the first 1200 integers is:  720600
\end{sphinxVerbatim}



\end{sphinxuseclass}
\end{sphinxuseclass}
}

\end{sphinxuseclass}
\end{sphinxuseclass}
\sphinxAtStartPar
Without getting into too many details, the code you just wrote starts initializing a variable S to zero, and then loops from 0 to 1200 assigning each time the value to a variable i, accumulating the sum of S + i in the variable S.

\sphinxAtStartPar
\sphinxstylestrong{A final thing to notice is indentation.}

\begin{sphinxadmonition}{note}{Info}\par
\sphinxAtStartPar
In Python it is important to indent the code properly as this provides the right scope for variables (e.g. see that the line \sphinxstyleemphasis{S = S + 1} starts more to the right than the previous and following line – this is because it is inside the for loop). You do not have to worry about this for the time being, we will get to this in a later practical…
\end{sphinxadmonition}

\sphinxAtStartPar
How does this code work? How does the value of S and i change as the code is executed? These are questions that can be answered by the debugger.

\sphinxAtStartPar
To start the debugger, click on \sphinxstylestrong{Debug –> Start Debugging} (shortcut F5). The following small panel should pop up:

\noindent\sphinxincludegraphics[width=213\sphinxpxdimen,height=53\sphinxpxdimen]{{debug}.png}

\sphinxAtStartPar
We will use it shortly, but before that, let’s focus on what we want to track. On the left hand side of the main panel, a \sphinxstyleemphasis{Watch} panel appeared. This is where we need to add the things we want to monitor as the execution of the program goes. With respect to the code written above, we are interested in keeping an eye on the variables S, i and also of the expression S+i (that will give us the value of S of the next iteration). Add these three expressions in the watch panel (click on \sphinxstylestrong{+} to add
new expressions). The watch panel should look like this:

\noindent\sphinxincludegraphics[width=256\sphinxpxdimen,height=100\sphinxpxdimen]{{watch}.png}

\sphinxAtStartPar
do not worry about the message “\sphinxstyleemphasis{name X is not defined}”, this is normal as no execution has taken place yet and the interpreter still does not know the value of these expressions.

\sphinxAtStartPar
The final thing before starting to debug is to set some breakpoints, places where the execution will stop so that we can check the value of the watched expressions. This can be done by hovering with the mouse on the left of the line number. A small reddish dot should appear, place the mouse over the correct line (e.g. the line corresponding to \sphinxstyleemphasis{S = S + 1} and click to add the breakpoint (a red dot should appear once you click).

\noindent\sphinxincludegraphics[width=565\sphinxpxdimen,height=216\sphinxpxdimen]{{breakpoint}.png}

\sphinxAtStartPar
Now we are ready to start debugging the code. Click on the green triangle on the small debug panel and you will see that the yellow arrow moved to the breakpoint and that the watch panel updated the value of all our expressions.

\noindent\sphinxincludegraphics[width=824\sphinxpxdimen,height=392\sphinxpxdimen]{{step0}.png}

\sphinxAtStartPar
The value of all expressions is zero because the debugger stopped \sphinxstylestrong{before} executing the code specified at the breakpoint line (recall that S is initialized to 0 and that i will range from 0 to 1200). If you click again on the green arrow, execution will continue until the next breakpoint (we are in a for loop, so this will be again the same line \sphinxhyphen{} trust me for the time being).

\noindent\sphinxincludegraphics[width=823\sphinxpxdimen,height=370\sphinxpxdimen]{{step1}.png}

\sphinxAtStartPar
Now i has been increased to 1, S is still 0 (remember that the execution stopped \sphinxstylestrong{before} executing the code at the breakpoint) and therefore S + i is now 1. Click one more time on the green arrow and values should update accordingly (i.e. S to 1, i to 2 and S + i to 3), another round of execution should update S to 3, i to 3 and S + i to 6. Got how this works? Variable i is increased by one each time, while S increases by i. You can go on for a few more iterations and see if this makes any
sense to you, once you are done with debugging you can stop the execution by pressing the red square on the small debug panel.

\begin{sphinxadmonition}{note}{Note}\par
\sphinxAtStartPar
The debugger is very useful to understand what your program does. Please spend some time to understand how this works as being able to run the debugger properly is a good help to identify and solve \sphinxstylestrong{semantic errors} of your code.
\end{sphinxadmonition}

\sphinxAtStartPar
Other editors are available, if you already have your favourite one you can stick to it. Some examples are:
\begin{itemize}
\item {} 
\sphinxAtStartPar
\sphinxhref{https://pythonhosted.org/spyder/installation.html}{Spyder}%
\begin{footnote}[7]\sphinxAtStartFootnote
\sphinxnolinkurl{https://pythonhosted.org/spyder/installation.html}
%
\end{footnote}

\item {} 
\sphinxAtStartPar
\sphinxhref{https://www.jetbrains.com/pycharm/}{PyCharm Community Edition}%
\begin{footnote}[8]\sphinxAtStartFootnote
\sphinxnolinkurl{https://www.jetbrains.com/pycharm/}
%
\end{footnote}

\item {} 
\sphinxAtStartPar
\sphinxhref{http://jupyter.org/}{Jupyter Notebook}%
\begin{footnote}[9]\sphinxAtStartFootnote
\sphinxnolinkurl{http://jupyter.org/}
%
\end{footnote}. Note: we might use it later on in the course.

\end{itemize}


\section{A quick Jupyter primer (just for your information, skip if not interested)}
\label{\detokenize{M1_practical1:A-quick-Jupyter-primer-(just-for-your-information,-skip-if-not-interested)}}
\sphinxAtStartPar
Jupyter allows to write notebooks organized in cells (these can be saved in files with .ipynb extension). Notebooks contain both the \sphinxstylestrong{code}, some \sphinxstylestrong{text describing the code} and the \sphinxstylestrong{output of the code execution}, they are quite useful to produce some quick reports on data analysis. where there is both code, output of running that code and text. The code by default is Python, but can also support other languages like R). The text is formatted using the \sphinxhref{https://en.wikipedia.org/wiki/Markdown}{Markdown
language}%
\begin{footnote}[10]\sphinxAtStartFootnote
\sphinxnolinkurl{https://en.wikipedia.org/wiki/Markdown}
%
\end{footnote} \sphinxhyphen{} see \sphinxhref{https://github.com/adam-p/markdown-here/wiki/Markdown-Cheatsheet}{cheatsheet}%
\begin{footnote}[11]\sphinxAtStartFootnote
\sphinxnolinkurl{https://github.com/adam-p/markdown-here/wiki/Markdown-Cheatsheet}
%
\end{footnote} for its details. \sphinxstyleemphasis{Jupyter is becoming the de\sphinxhyphen{}facto standard for writing technical documentation}.


\subsection{Installation}
\label{\detokenize{M1_practical1:Installation}}
\sphinxAtStartPar
To install it (if you have not installed python with Anaconda otherwise you should have it already):

\begin{sphinxVerbatim}[commandchars=\\\{\}]
python3 \PYGZhy{}m pip install jupyter
\end{sphinxVerbatim}

\sphinxAtStartPar
you can find more information \sphinxhref{https://jupyter.org/install}{here}%
\begin{footnote}[12]\sphinxAtStartFootnote
\sphinxnolinkurl{https://jupyter.org/install}
%
\end{footnote}

\sphinxAtStartPar
Upon successful installation, you can run it with:

\begin{sphinxVerbatim}[commandchars=\\\{\}]
jupyter\PYGZhy{}notebook
\end{sphinxVerbatim}

\sphinxAtStartPar
This should fire up a browser on a page where you can start creating your notebooks or modifying existing ones. To create a new notebook you simply click on \sphinxstylestrong{New}:

\noindent\sphinxincludegraphics[width=1171\sphinxpxdimen,height=292\sphinxpxdimen]{{jupyter}.png}

\sphinxAtStartPar
and then you can start adding cells (i.e. containers of code and text). The type of each cell is specified by selecting the cell and selecting the right type in the dropdown list:

\noindent\sphinxincludegraphics[width=1164\sphinxpxdimen,height=399\sphinxpxdimen]{{juptyter2b}.png}

\sphinxAtStartPar
Cells can be executed by clicking on the \sphinxstylestrong{Run} button. This will get the code to execute (and output to be written) and text to be processed to provide the final page layout. To go back to the edit mode, just double click on an executed cell.

\noindent\sphinxincludegraphics[width=1184\sphinxpxdimen,height=447\sphinxpxdimen]{{jupyter3}.png}

\sphinxAtStartPar
\sphinxstyleemphasis{Please take some more time to familiarize with Visual Studio Code (creating files, saving files, interacting with the debugger etc.) as in the next practicals we will take this ability for granted. Once you are done you can move on and do the following exercises.}


\section{Let’s try together}
\label{\detokenize{M1_practical1:Let's-try-together}}\begin{enumerate}
\sphinxsetlistlabels{\arabic}{enumi}{enumii}{}{.}%
\item {} 
\sphinxAtStartPar
Compute the area of a triangle having base 120 units (B) and height 33 (H). Assign the result to a variable named area and print it.

\end{enumerate}



\sphinxAtStartPar
Show/Hide Solution





\begin{sphinxuseclass}{nbinput}
{
\begin{sphinxVerbatim}[commandchars=\\\{\}]
\llap{\color{nbsphinxin}[17]:\,\hspace{\fboxrule}\hspace{\fboxsep}}\PYG{n}{B} \PYG{o}{=} \PYG{l+m+mi}{120}
\PYG{n}{H} \PYG{o}{=} \PYG{l+m+mi}{33}
\PYG{n}{Area} \PYG{o}{=} \PYG{n}{B}\PYG{o}{*}\PYG{n}{H}\PYG{o}{/}\PYG{l+m+mi}{2}
\PYG{n+nb}{print}\PYG{p}{(}\PYG{l+s+s2}{\PYGZdq{}}\PYG{l+s+s2}{Triangle area is:}\PYG{l+s+s2}{\PYGZdq{}}\PYG{p}{,} \PYG{n}{Area}\PYG{p}{)}
\end{sphinxVerbatim}
}

\end{sphinxuseclass}
\begin{sphinxuseclass}{nboutput}
\begin{sphinxuseclass}{nblast}
{

\kern-\sphinxverbatimsmallskipamount\kern-\baselineskip
\kern+\FrameHeightAdjust\kern-\fboxrule
\vspace{\nbsphinxcodecellspacing}

\sphinxsetup{VerbatimColor={named}{white}}
\begin{sphinxuseclass}{output_area}
\begin{sphinxuseclass}{}


\begin{sphinxVerbatim}[commandchars=\\\{\}]
Triangle area is: 1980.0
\end{sphinxVerbatim}



\end{sphinxuseclass}
\end{sphinxuseclass}
}

\end{sphinxuseclass}
\end{sphinxuseclass}

\begin{enumerate}
\sphinxsetlistlabels{\arabic}{enumi}{enumii}{}{.}%
\setcounter{enumi}{1}
\item {} 
\sphinxAtStartPar
Aquire the side S (base of a square) from the user at runtime and compute the area of a square. Then, assign the result to a variable named area and print it.
Hint: use the input function (details \sphinxhref{https://docs.python.org/3/library/functions.html\#input}{here}%
\begin{footnote}[13]\sphinxAtStartFootnote
\sphinxnolinkurl{https://docs.python.org/3/library/functions.html\#input}
%
\end{footnote}) and remember to convert the acquired value into an int.

\end{enumerate}



\sphinxAtStartPar
Show/Hide Solution





\begin{sphinxuseclass}{nbinput}
{
\begin{sphinxVerbatim}[commandchars=\\\{\}]
\llap{\color{nbsphinxin}[19]:\,\hspace{\fboxrule}\hspace{\fboxsep}}\PYG{n}{S\PYGZus{}str} \PYG{o}{=} \PYG{n+nb}{input}\PYG{p}{(}\PYG{l+s+s2}{\PYGZdq{}}\PYG{l+s+s2}{Insert size: }\PYG{l+s+s2}{\PYGZdq{}}\PYG{p}{)}
\PYG{n+nb}{print}\PYG{p}{(}\PYG{n+nb}{type}\PYG{p}{(}\PYG{n}{S\PYGZus{}str}\PYG{p}{)}\PYG{p}{)}
\PYG{n+nb}{print}\PYG{p}{(}\PYG{n}{S\PYGZus{}str}\PYG{p}{)}
\PYG{n}{S} \PYG{o}{=} \PYG{n+nb}{int}\PYG{p}{(}\PYG{n}{S\PYGZus{}str}\PYG{p}{)}
\PYG{n+nb}{print}\PYG{p}{(}\PYG{n+nb}{type}\PYG{p}{(}\PYG{n}{S}\PYG{p}{)}\PYG{p}{)}
\PYG{n+nb}{print}\PYG{p}{(}\PYG{n}{S}\PYG{p}{)}
\PYG{n}{Area} \PYG{o}{=} \PYG{n}{S}\PYG{o}{*}\PYG{o}{*}\PYG{l+m+mi}{2}
\PYG{n+nb}{print}\PYG{p}{(}\PYG{l+s+s2}{\PYGZdq{}}\PYG{l+s+s2}{Square area is:}\PYG{l+s+s2}{\PYGZdq{}}\PYG{p}{,}\PYG{n}{Area}\PYG{p}{)}
\end{sphinxVerbatim}
}

\end{sphinxuseclass}
\begin{sphinxuseclass}{nboutput}
\begin{sphinxuseclass}{nblast}
{

\kern-\sphinxverbatimsmallskipamount\kern-\baselineskip
\kern+\FrameHeightAdjust\kern-\fboxrule
\vspace{\nbsphinxcodecellspacing}

\sphinxsetup{VerbatimColor={named}{white}}
\begin{sphinxuseclass}{output_area}
\begin{sphinxuseclass}{}


\begin{sphinxVerbatim}[commandchars=\\\{\}]
<class 'str'>
5
<class 'int'>
5
Square area is: 25
\end{sphinxVerbatim}



\end{sphinxuseclass}
\end{sphinxuseclass}
}

\end{sphinxuseclass}
\end{sphinxuseclass}

\section{Exercises}
\label{\detokenize{M1_practical1:Exercises}}\begin{enumerate}
\sphinxsetlistlabels{\arabic}{enumi}{enumii}{}{.}%
\item {} 
\sphinxAtStartPar
If you have not done so already, put the two previous scripts in two separate files (e.g. triangle\_area.py and square\_area.py and execute them from the terminal).

\item {} 
\sphinxAtStartPar
Write a small script (trapezoid.py) that computes the area of a trapezoid having major base (MB) equal to 30 units, minor base (mb) equal to 12 and height (H) equal to 17. Print the resulting area. Try executing the script from inside Visual Studio Code and from the terminal.

\end{enumerate}



\sphinxAtStartPar
Show/Hide Solution





\begin{sphinxuseclass}{nbinput}
{
\begin{sphinxVerbatim}[commandchars=\\\{\}]
\llap{\color{nbsphinxin}[5]:\,\hspace{\fboxrule}\hspace{\fboxsep}}\PYG{l+s+sd}{\PYGZdq{}\PYGZdq{}\PYGZdq{}trapezoid.py\PYGZdq{}\PYGZdq{}\PYGZdq{}}
\PYG{n}{MB} \PYG{o}{=} \PYG{l+m+mi}{30}
\PYG{n}{mb} \PYG{o}{=} \PYG{l+m+mi}{12}
\PYG{n}{H} \PYG{o}{=} \PYG{l+m+mi}{17}
\PYG{n}{Area} \PYG{o}{=} \PYG{p}{(}\PYG{n}{MB} \PYG{o}{+} \PYG{n}{mb}\PYG{p}{)}\PYG{o}{*}\PYG{n}{H}\PYG{o}{/}\PYG{l+m+mi}{2}
\PYG{n+nb}{print}\PYG{p}{(}\PYG{l+s+s2}{\PYGZdq{}}\PYG{l+s+s2}{Trapezoid area is: }\PYG{l+s+s2}{\PYGZdq{}}\PYG{p}{,} \PYG{n}{Area}\PYG{p}{)}
\end{sphinxVerbatim}
}

\end{sphinxuseclass}
\begin{sphinxuseclass}{nboutput}
\begin{sphinxuseclass}{nblast}
{

\kern-\sphinxverbatimsmallskipamount\kern-\baselineskip
\kern+\FrameHeightAdjust\kern-\fboxrule
\vspace{\nbsphinxcodecellspacing}

\sphinxsetup{VerbatimColor={named}{white}}
\begin{sphinxuseclass}{output_area}
\begin{sphinxuseclass}{}


\begin{sphinxVerbatim}[commandchars=\\\{\}]
Trapezoid area is:  357.0
\end{sphinxVerbatim}



\end{sphinxuseclass}
\end{sphinxuseclass}
}

\end{sphinxuseclass}
\end{sphinxuseclass}

\begin{enumerate}
\sphinxsetlistlabels{\arabic}{enumi}{enumii}{}{.}%
\setcounter{enumi}{2}
\item {} 
\sphinxAtStartPar
Write a program to compute the sum of the first 1200 integers by using the following equation: \(\sum\limits_{i=1}^n i = \frac{n (n+1)}{2}\).

\end{enumerate}



\sphinxAtStartPar
Show/Hide Solution





\begin{sphinxuseclass}{nbinput}
{
\begin{sphinxVerbatim}[commandchars=\\\{\}]
\llap{\color{nbsphinxin}[21]:\,\hspace{\fboxrule}\hspace{\fboxsep}}\PYG{n}{N} \PYG{o}{=} \PYG{l+m+mi}{1200}

\PYG{n+nb}{print}\PYG{p}{(}\PYG{l+s+s2}{\PYGZdq{}}\PYG{l+s+s2}{Sum of first 1200 integers: }\PYG{l+s+s2}{\PYGZdq{}}\PYG{p}{,} \PYG{n}{N}\PYG{o}{*}\PYG{p}{(}\PYG{n}{N}\PYG{o}{+}\PYG{l+m+mi}{1}\PYG{p}{)}\PYG{o}{/}\PYG{l+m+mi}{2}\PYG{p}{)}
\end{sphinxVerbatim}
}

\end{sphinxuseclass}
\begin{sphinxuseclass}{nboutput}
\begin{sphinxuseclass}{nblast}
{

\kern-\sphinxverbatimsmallskipamount\kern-\baselineskip
\kern+\FrameHeightAdjust\kern-\fboxrule
\vspace{\nbsphinxcodecellspacing}

\sphinxsetup{VerbatimColor={named}{white}}
\begin{sphinxuseclass}{output_area}
\begin{sphinxuseclass}{}


\begin{sphinxVerbatim}[commandchars=\\\{\}]
Sum of first 1200 integers:  720600.0
\end{sphinxVerbatim}



\end{sphinxuseclass}
\end{sphinxuseclass}
}

\end{sphinxuseclass}
\end{sphinxuseclass}

\begin{enumerate}
\sphinxsetlistlabels{\arabic}{enumi}{enumii}{}{.}%
\setcounter{enumi}{3}
\item {} 
\sphinxAtStartPar
Modify the program at point 3. to make it acquire the number of integers to sum N from the user at runtime.

\end{enumerate}



\sphinxAtStartPar
Show/Hide Solution





\begin{sphinxuseclass}{nbinput}
{
\begin{sphinxVerbatim}[commandchars=\\\{\}]
\llap{\color{nbsphinxin}[23]:\,\hspace{\fboxrule}\hspace{\fboxsep}}\PYG{n+nb}{print}\PYG{p}{(}\PYG{l+s+s2}{\PYGZdq{}}\PYG{l+s+s2}{Input number N:}\PYG{l+s+s2}{\PYGZdq{}}\PYG{p}{)}
\PYG{n}{N} \PYG{o}{=} \PYG{n+nb}{int}\PYG{p}{(}\PYG{n+nb}{input}\PYG{p}{(}\PYG{p}{)}\PYG{p}{)}
\PYG{n+nb}{print}\PYG{p}{(}\PYG{l+s+s2}{\PYGZdq{}}\PYG{l+s+s2}{Sum of first }\PYG{l+s+s2}{\PYGZdq{}}\PYG{p}{,} \PYG{n}{N}\PYG{p}{,} \PYG{l+s+s2}{\PYGZdq{}}\PYG{l+s+s2}{ integers: }\PYG{l+s+s2}{\PYGZdq{}}\PYG{p}{,} \PYG{n}{N}\PYG{o}{*}\PYG{p}{(}\PYG{n}{N}\PYG{o}{+}\PYG{l+m+mi}{1}\PYG{p}{)}\PYG{o}{/}\PYG{l+m+mi}{2}\PYG{p}{)}
\end{sphinxVerbatim}
}

\end{sphinxuseclass}
\begin{sphinxuseclass}{nboutput}
\begin{sphinxuseclass}{nblast}
{

\kern-\sphinxverbatimsmallskipamount\kern-\baselineskip
\kern+\FrameHeightAdjust\kern-\fboxrule
\vspace{\nbsphinxcodecellspacing}

\sphinxsetup{VerbatimColor={named}{white}}
\begin{sphinxuseclass}{output_area}
\begin{sphinxuseclass}{}


\begin{sphinxVerbatim}[commandchars=\\\{\}]
Input number N:
Sum of first  6  integers:  21.0
\end{sphinxVerbatim}



\end{sphinxuseclass}
\end{sphinxuseclass}
}

\end{sphinxuseclass}
\end{sphinxuseclass}

\begin{enumerate}
\sphinxsetlistlabels{\arabic}{enumi}{enumii}{}{.}%
\setcounter{enumi}{4}
\item {} 
\sphinxAtStartPar
Write a small script to compute the length of the hypotenuse (c) of a right triangle having sides a=133 and b=72 units (see picture below). Hint: \sphinxstyleemphasis{remember the Pythagorean theorem and use math.sqrt}.

\end{enumerate}

\noindent\sphinxincludegraphics[width=220\sphinxpxdimen,height=178\sphinxpxdimen]{{triangle}.png}



\sphinxAtStartPar
Show/Hide Solution





\begin{sphinxuseclass}{nbinput}
{
\begin{sphinxVerbatim}[commandchars=\\\{\}]
\llap{\color{nbsphinxin}[24]:\,\hspace{\fboxrule}\hspace{\fboxsep}}\PYG{k+kn}{import} \PYG{n+nn}{math}

\PYG{n}{a} \PYG{o}{=} \PYG{l+m+mi}{133}
\PYG{n}{b} \PYG{o}{=} \PYG{l+m+mi}{72}

\PYG{n}{c} \PYG{o}{=} \PYG{n}{math}\PYG{o}{.}\PYG{n}{sqrt}\PYG{p}{(}\PYG{n}{a}\PYG{o}{*}\PYG{o}{*}\PYG{l+m+mi}{2} \PYG{o}{+} \PYG{n}{b}\PYG{o}{*}\PYG{o}{*}\PYG{l+m+mi}{2}\PYG{p}{)}

\PYG{n+nb}{print}\PYG{p}{(}\PYG{l+s+s2}{\PYGZdq{}}\PYG{l+s+s2}{Hypotenuse: }\PYG{l+s+s2}{\PYGZdq{}}\PYG{p}{,} \PYG{n}{c}\PYG{p}{)}
\end{sphinxVerbatim}
}

\end{sphinxuseclass}
\begin{sphinxuseclass}{nboutput}
\begin{sphinxuseclass}{nblast}
{

\kern-\sphinxverbatimsmallskipamount\kern-\baselineskip
\kern+\FrameHeightAdjust\kern-\fboxrule
\vspace{\nbsphinxcodecellspacing}

\sphinxsetup{VerbatimColor={named}{white}}
\begin{sphinxuseclass}{output_area}
\begin{sphinxuseclass}{}


\begin{sphinxVerbatim}[commandchars=\\\{\}]
Hypotenuse:  151.23822268196622
\end{sphinxVerbatim}



\end{sphinxuseclass}
\end{sphinxuseclass}
}

\end{sphinxuseclass}
\end{sphinxuseclass}

\begin{enumerate}
\sphinxsetlistlabels{\arabic}{enumi}{enumii}{}{.}%
\setcounter{enumi}{5}
\item {} 
\sphinxAtStartPar
Rewrite the trapezoid script making it compute the area of the trapezoid starting from the major base (MB), minor base (mb) and height (H) taken in input. (Hint: \sphinxstyleemphasis{use the input function and remember to convert the acquired value into an int}).

\end{enumerate}



\sphinxAtStartPar
Show/Hide Solution





\begin{sphinxuseclass}{nbinput}
{
\begin{sphinxVerbatim}[commandchars=\\\{\}]
\llap{\color{nbsphinxin}[25]:\,\hspace{\fboxrule}\hspace{\fboxsep}}\PYG{l+s+sd}{\PYGZdq{}\PYGZdq{}\PYGZdq{}trapezoidV2.py\PYGZdq{}\PYGZdq{}\PYGZdq{}}
\PYG{n}{MB} \PYG{o}{=} \PYG{n+nb}{int}\PYG{p}{(}\PYG{n+nb}{input}\PYG{p}{(}\PYG{l+s+s2}{\PYGZdq{}}\PYG{l+s+s2}{Input the major base (MB):}\PYG{l+s+s2}{\PYGZdq{}}\PYG{p}{)}\PYG{p}{)}
\PYG{n}{mb} \PYG{o}{=} \PYG{n+nb}{int}\PYG{p}{(}\PYG{n+nb}{input}\PYG{p}{(}\PYG{l+s+s2}{\PYGZdq{}}\PYG{l+s+s2}{Input the minor base (mb):}\PYG{l+s+s2}{\PYGZdq{}}\PYG{p}{)}\PYG{p}{)}
\PYG{n}{H} \PYG{o}{=} \PYG{n+nb}{int}\PYG{p}{(}\PYG{n+nb}{input}\PYG{p}{(}\PYG{l+s+s2}{\PYGZdq{}}\PYG{l+s+s2}{Input the height (H):}\PYG{l+s+s2}{\PYGZdq{}}\PYG{p}{)}\PYG{p}{)}
\PYG{n}{Area} \PYG{o}{=} \PYG{p}{(}\PYG{n}{MB} \PYG{o}{+} \PYG{n}{mb}\PYG{p}{)}\PYG{o}{*}\PYG{n}{H}\PYG{o}{/}\PYG{l+m+mi}{2}
\PYG{n+nb}{print}\PYG{p}{(}\PYG{l+s+s2}{\PYGZdq{}}\PYG{l+s+s2}{Given MB:}\PYG{l+s+s2}{\PYGZdq{}}\PYG{p}{,} \PYG{n+nb}{str}\PYG{p}{(}\PYG{n}{MB}\PYG{p}{)} \PYG{p}{,} \PYG{l+s+s2}{\PYGZdq{}}\PYG{l+s+s2}{ mb:}\PYG{l+s+s2}{\PYGZdq{}}\PYG{p}{,} \PYG{n+nb}{str}\PYG{p}{(}\PYG{n}{mb}\PYG{p}{)} \PYG{p}{,} \PYG{l+s+s2}{\PYGZdq{}}\PYG{l+s+s2}{ and H:}\PYG{l+s+s2}{\PYGZdq{}}\PYG{p}{,} \PYG{n}{H}\PYG{p}{)}
\PYG{n+nb}{print}\PYG{p}{(}\PYG{l+s+s2}{\PYGZdq{}}\PYG{l+s+s2}{The trapezoid area is: }\PYG{l+s+s2}{\PYGZdq{}}\PYG{p}{,} \PYG{n}{Area}\PYG{p}{)}
\end{sphinxVerbatim}
}

\end{sphinxuseclass}
\begin{sphinxuseclass}{nboutput}
\begin{sphinxuseclass}{nblast}
{

\kern-\sphinxverbatimsmallskipamount\kern-\baselineskip
\kern+\FrameHeightAdjust\kern-\fboxrule
\vspace{\nbsphinxcodecellspacing}

\sphinxsetup{VerbatimColor={named}{white}}
\begin{sphinxuseclass}{output_area}
\begin{sphinxuseclass}{}


\begin{sphinxVerbatim}[commandchars=\\\{\}]
Given MB: 12  mb: 55  and H: 66
The trapezoid area is:  2211.0
\end{sphinxVerbatim}



\end{sphinxuseclass}
\end{sphinxuseclass}
}

\end{sphinxuseclass}
\end{sphinxuseclass}

\begin{enumerate}
\sphinxsetlistlabels{\arabic}{enumi}{enumii}{}{.}%
\setcounter{enumi}{6}
\item {} 
\sphinxAtStartPar
Write a script that reads the side of an hexagon in input and computes its perimeter and area printing them to the screen. \sphinxstyleemphasis{Hint:} \(Area = \frac{3*\sqrt{3}*side^{2}}{2}\)

\end{enumerate}



\sphinxAtStartPar
Show/Hide Solution





\begin{sphinxuseclass}{nbinput}
{
\begin{sphinxVerbatim}[commandchars=\\\{\}]
\llap{\color{nbsphinxin}[26]:\,\hspace{\fboxrule}\hspace{\fboxsep}}\PYG{k+kn}{import} \PYG{n+nn}{math}

\PYG{n}{side} \PYG{o}{=} \PYG{n+nb}{int}\PYG{p}{(}\PYG{n+nb}{input}\PYG{p}{(}\PYG{l+s+s2}{\PYGZdq{}}\PYG{l+s+s2}{Please insert the side of the hexagon: }\PYG{l+s+s2}{\PYGZdq{}}\PYG{p}{)}\PYG{p}{)}

\PYG{n}{P} \PYG{o}{=} \PYG{l+m+mi}{6}\PYG{o}{*}\PYG{n}{side}
\PYG{n}{A} \PYG{o}{=} \PYG{p}{(}\PYG{l+m+mi}{3}\PYG{o}{*}\PYG{n}{math}\PYG{o}{.}\PYG{n}{sqrt}\PYG{p}{(}\PYG{l+m+mi}{3}\PYG{p}{)}\PYG{o}{*}\PYG{n}{side}\PYG{o}{*}\PYG{o}{*}\PYG{l+m+mi}{2}\PYG{p}{)}\PYG{o}{/}\PYG{l+m+mi}{2}
\PYG{n+nb}{print}\PYG{p}{(}\PYG{l+s+s2}{\PYGZdq{}}\PYG{l+s+s2}{Perimeter: }\PYG{l+s+s2}{\PYGZdq{}}\PYG{p}{,} \PYG{n}{P}\PYG{p}{,} \PYG{l+s+s2}{\PYGZdq{}}\PYG{l+s+s2}{ Area: }\PYG{l+s+s2}{\PYGZdq{}}\PYG{p}{,} \PYG{n}{A}\PYG{p}{)}
\end{sphinxVerbatim}
}

\end{sphinxuseclass}
\begin{sphinxuseclass}{nboutput}
\begin{sphinxuseclass}{nblast}
{

\kern-\sphinxverbatimsmallskipamount\kern-\baselineskip
\kern+\FrameHeightAdjust\kern-\fboxrule
\vspace{\nbsphinxcodecellspacing}

\sphinxsetup{VerbatimColor={named}{white}}
\begin{sphinxuseclass}{output_area}
\begin{sphinxuseclass}{}


\begin{sphinxVerbatim}[commandchars=\\\{\}]
Perimeter:  3330  Area:  800272.4250021052
\end{sphinxVerbatim}



\end{sphinxuseclass}
\end{sphinxuseclass}
}

\end{sphinxuseclass}
\end{sphinxuseclass}

\begin{enumerate}
\sphinxsetlistlabels{\arabic}{enumi}{enumii}{}{.}%
\setcounter{enumi}{7}
\item {} 
\sphinxAtStartPar
You are given a very important deadline in: days = 4 hours = 13 minutes = 52
Write some code that prints the total minutes.

\end{enumerate}



\sphinxAtStartPar
Show/Hide Solution





\begin{sphinxuseclass}{nbinput}
{
\begin{sphinxVerbatim}[commandchars=\\\{\}]
\llap{\color{nbsphinxin}[27]:\,\hspace{\fboxrule}\hspace{\fboxsep}}\PYG{n}{days} \PYG{o}{=} \PYG{l+m+mi}{4}
\PYG{n}{hours} \PYG{o}{=} \PYG{l+m+mi}{13}
\PYG{n}{minutes} \PYG{o}{=} \PYG{l+m+mi}{52}

\PYG{c+c1}{\PYGZsh{} write here}
\PYG{n+nb}{print}\PYG{p}{(}\PYG{l+s+s2}{\PYGZdq{}}\PYG{l+s+s2}{In total there are}\PYG{l+s+s2}{\PYGZdq{}}\PYG{p}{,} \PYG{n}{days}\PYG{o}{*}\PYG{l+m+mi}{24}\PYG{o}{*}\PYG{l+m+mi}{60} \PYG{o}{+} \PYG{n}{hours}\PYG{o}{*}\PYG{l+m+mi}{60} \PYG{o}{+} \PYG{n}{minutes}\PYG{p}{,} \PYG{l+s+s2}{\PYGZdq{}}\PYG{l+s+s2}{minutes left}\PYG{l+s+s2}{\PYGZdq{}}\PYG{p}{)}
\end{sphinxVerbatim}
}

\end{sphinxuseclass}
\begin{sphinxuseclass}{nboutput}
\begin{sphinxuseclass}{nblast}
{

\kern-\sphinxverbatimsmallskipamount\kern-\baselineskip
\kern+\FrameHeightAdjust\kern-\fboxrule
\vspace{\nbsphinxcodecellspacing}

\sphinxsetup{VerbatimColor={named}{white}}
\begin{sphinxuseclass}{output_area}
\begin{sphinxuseclass}{}


\begin{sphinxVerbatim}[commandchars=\\\{\}]
In total there are 6592 minutes left
\end{sphinxVerbatim}



\end{sphinxuseclass}
\end{sphinxuseclass}
}

\end{sphinxuseclass}
\end{sphinxuseclass}

\section{Exercises (Difficult)}
\label{\detokenize{M1_practical1:Exercises-(Difficult)}}\begin{enumerate}
\sphinxsetlistlabels{\arabic}{enumi}{enumii}{}{.}%
\item {} 
\sphinxAtStartPar
\sphinxstylestrong{Difficult} Write a function is\_balanced(word) that: 1)takes a string of lowercase letters, 2)returns True iff for every letter the number of occurrences equals the number of occurrences of its opposite in the alphabet (a↔z, b↔y, …). The empty string counts as balanced.

\end{enumerate}

\begin{sphinxVerbatim}[commandchars=\\\{\}]
is\PYGZus{}balanced(\PYGZdq{}azby\PYGZdq{})   \PYGZsh{} True
is\PYGZus{}balanced(\PYGZdq{}aazz\PYGZdq{})   \PYGZsh{} True
is\PYGZus{}balanced(\PYGZdq{}abc\PYGZdq{})    \PYGZsh{} False
is\PYGZus{}balanced(\PYGZdq{}\PYGZdq{})       \PYGZsh{} True
\end{sphinxVerbatim}



\sphinxAtStartPar
Show/Hide Solution





\begin{sphinxuseclass}{nbinput}
\begin{sphinxuseclass}{nblast}
{
\begin{sphinxVerbatim}[commandchars=\\\{\}]
\llap{\color{nbsphinxin}[3]:\,\hspace{\fboxrule}\hspace{\fboxsep}}\PYG{k}{def} \PYG{n+nf}{is\PYGZus{}balanced}\PYG{p}{(}\PYG{n}{word}\PYG{p}{:} \PYG{n+nb}{str}\PYG{p}{)} \PYG{o}{\PYGZhy{}}\PYG{o}{\PYGZgt{}} \PYG{n+nb}{bool}\PYG{p}{:}
    \PYG{c+c1}{\PYGZsh{} frequency array for \PYGZsq{}a\PYGZsq{}..\PYGZsq{}z\PYGZsq{}}
    \PYG{n}{freq} \PYG{o}{=} \PYG{p}{[}\PYG{l+m+mi}{0}\PYG{p}{]} \PYG{o}{*} \PYG{l+m+mi}{26}
    \PYG{k}{for} \PYG{n}{ch} \PYG{o+ow}{in} \PYG{n}{word}\PYG{p}{:}
        \PYG{c+c1}{\PYGZsh{} assume input is lowercase a\PYGZhy{}z as stated}
        \PYG{n}{freq}\PYG{p}{[}\PYG{n+nb}{ord}\PYG{p}{(}\PYG{n}{ch}\PYG{p}{)} \PYG{o}{\PYGZhy{}} \PYG{n+nb}{ord}\PYG{p}{(}\PYG{l+s+s1}{\PYGZsq{}}\PYG{l+s+s1}{a}\PYG{l+s+s1}{\PYGZsq{}}\PYG{p}{)}\PYG{p}{]} \PYG{o}{+}\PYG{o}{=} \PYG{l+m+mi}{1}

    \PYG{c+c1}{\PYGZsh{} check symmetry: i \PYGZlt{}\PYGZhy{}\PYGZgt{} 25 \PYGZhy{} i}
    \PYG{k}{for} \PYG{n}{i} \PYG{o+ow}{in} \PYG{n+nb}{range}\PYG{p}{(}\PYG{l+m+mi}{26}\PYG{p}{)}\PYG{p}{:}
        \PYG{k}{if} \PYG{n}{freq}\PYG{p}{[}\PYG{n}{i}\PYG{p}{]} \PYG{o}{!=} \PYG{n}{freq}\PYG{p}{[}\PYG{l+m+mi}{25} \PYG{o}{\PYGZhy{}} \PYG{n}{i}\PYG{p}{]}\PYG{p}{:}
            \PYG{k}{return} \PYG{k+kc}{False}
    \PYG{k}{return} \PYG{k+kc}{True}

\end{sphinxVerbatim}
}

\end{sphinxuseclass}
\end{sphinxuseclass}

\begin{enumerate}
\sphinxsetlistlabels{\arabic}{enumi}{enumii}{}{.}%
\setcounter{enumi}{1}
\item {} 
\sphinxAtStartPar
\sphinxstylestrong{Difficult} Write a function named is\_magic\_square(matrix), that takes a square matrix of integers (list of lists) and returns True if the sum of every row and every column is the same number, otherwise False. (We do not require diagonal sums to match; we only check rows and columns.)

\end{enumerate}

\sphinxAtStartPar
Examples

\begin{sphinxVerbatim}[commandchars=\\\{\}]
m1 = [
    [2, 2],
    [2, 2]
]
\PYGZsh{} all rows and columns sum to 4 → True
print(is\PYGZus{}magic\PYGZus{}square(m1))   \PYGZsh{} True

m2 = [
    [1, 2],
    [3, 4]
]
print(is\PYGZus{}magic\PYGZus{}square(m2))   \PYGZsh{} False
\end{sphinxVerbatim}



\sphinxAtStartPar
Show/Hide Solution





\begin{sphinxuseclass}{nbinput}
\begin{sphinxuseclass}{nblast}
{
\begin{sphinxVerbatim}[commandchars=\\\{\}]
\llap{\color{nbsphinxin}[6]:\,\hspace{\fboxrule}\hspace{\fboxsep}}\PYG{k}{def} \PYG{n+nf}{is\PYGZus{}magic\PYGZus{}square}\PYG{p}{(}\PYG{n}{matrix}\PYG{p}{)}\PYG{p}{:}
    \PYG{n}{n} \PYG{o}{=} \PYG{n+nb}{len}\PYG{p}{(}\PYG{n}{matrix}\PYG{p}{)}
    \PYG{n}{target} \PYG{o}{=} \PYG{n+nb}{sum}\PYG{p}{(}\PYG{n}{matrix}\PYG{p}{[}\PYG{l+m+mi}{0}\PYG{p}{]}\PYG{p}{)}  \PYG{c+c1}{\PYGZsh{} sum of first row as reference}

    \PYG{c+c1}{\PYGZsh{} check all rows}
    \PYG{k}{for} \PYG{n}{row} \PYG{o+ow}{in} \PYG{n}{matrix}\PYG{p}{:}
        \PYG{k}{if} \PYG{n+nb}{sum}\PYG{p}{(}\PYG{n}{row}\PYG{p}{)} \PYG{o}{!=} \PYG{n}{target}\PYG{p}{:}
            \PYG{k}{return} \PYG{k+kc}{False}

    \PYG{c+c1}{\PYGZsh{} check all columns}
    \PYG{k}{for} \PYG{n}{c} \PYG{o+ow}{in} \PYG{n+nb}{range}\PYG{p}{(}\PYG{n}{n}\PYG{p}{)}\PYG{p}{:}
        \PYG{k}{if} \PYG{n+nb}{sum}\PYG{p}{(}\PYG{n}{matrix}\PYG{p}{[}\PYG{n}{r}\PYG{p}{]}\PYG{p}{[}\PYG{n}{c}\PYG{p}{]} \PYG{k}{for} \PYG{n}{r} \PYG{o+ow}{in} \PYG{n+nb}{range}\PYG{p}{(}\PYG{n}{n}\PYG{p}{)}\PYG{p}{)} \PYG{o}{!=} \PYG{n}{target}\PYG{p}{:}
            \PYG{k}{return} \PYG{k+kc}{False}

    \PYG{k}{return} \PYG{k+kc}{True}
\end{sphinxVerbatim}
}

\end{sphinxuseclass}
\end{sphinxuseclass}



\section{Let’s try Linux}
\label{\detokenize{M1_practical1:Let's-try-Linux}}
\begin{sphinxadmonition}{note}{Note:}\par
\sphinxAtStartPar
Altough for this course you will be fine with any operating system, my advice is to get familiar with Linux.
\end{sphinxadmonition}

\sphinxAtStartPar
The following section explains how to install Linux on a windows machine. This is for your reference, you can read the following instructions before the next practical and try to instally Linux if you want to test it out.


\subsection{Linux on windows}
\label{\detokenize{M1_practical1:Linux-on-windows}}
\sphinxAtStartPar
If your computer has Windows installed but you want to learn Linux you have several options to get it to run Linux:
\begin{enumerate}
\sphinxsetlistlabels{\arabic}{enumi}{enumii}{}{.}%
\item {} 
\sphinxAtStartPar
You can install a virtualization software like \sphinxhref{https://my.vmware.com/en/web/vmware/free\#desktop\_end\_user\_computing/vmware\_workstation\_player/15\_0\%7CPLAYER-1550\%7Cproduct\_downloads}{vmware player}%
\begin{footnote}[14]\sphinxAtStartFootnote
\sphinxnolinkurl{https://my.vmware.com/en/web/vmware/free\#desktop\_end\_user\_computing/vmware\_workstation\_player/15\_0\%7CPLAYER-1550\%7Cproduct\_downloads}
%
\end{footnote} and download the .iso image of a linux distribution like \sphinxhref{https://ubuntu.com/\#download}{ubuntu.}%
\begin{footnote}[15]\sphinxAtStartFootnote
\sphinxnolinkurl{https://ubuntu.com/\#download}
%
\end{footnote} and install/run it from vmware player. For more information you can look at \sphinxhref{https://www.youtube.com/watch?v=9rUhGWijf9U}{this tutorial.}%
\begin{footnote}[16]\sphinxAtStartFootnote
\sphinxnolinkurl{https://www.youtube.com/watch?v=9rUhGWijf9U}
%
\end{footnote} Another option is to install \sphinxhref{https://www.virtualbox.org/wiki/Downloads}{virtual
box}%
\begin{footnote}[17]\sphinxAtStartFootnote
\sphinxnolinkurl{https://www.virtualbox.org/wiki/Downloads}
%
\end{footnote}.

\item {} 
\sphinxAtStartPar
This video tutorial (only in Italian) shows you how to set up a usb stick to run Linux from it: \sphinxurl{https://youtu.be/8\_SK8iEMyJk}

\end{enumerate}

\sphinxAtStartPar
\sphinxhref{https://www.osboxes.org/virtualbox-images/}{Here}%
\begin{footnote}[18]\sphinxAtStartFootnote
\sphinxnolinkurl{https://www.osboxes.org/virtualbox-images/}
%
\end{footnote} you can find some VDI images that you can load in virtual box or in vmware player with several different operating systems including Linux distributions like Ubuntu, Debian, Centos, Fedora, etc. Please refer to this \sphinxhref{https://www.osboxes.org/guide/}{guide}%
\begin{footnote}[19]\sphinxAtStartFootnote
\sphinxnolinkurl{https://www.osboxes.org/guide/}
%
\end{footnote} (for information on vmware please click on \sphinxstylestrong{VM IMAGES –> VMware IMAGES} in the menu of the page).


\subsection{A dual boot system}
\label{\detokenize{M1_practical1:A-dual-boot-system}}
\sphinxAtStartPar
You can also install \sphinxstylestrong{Linux and Windows on the same machine} and every time you boot your system up \sphinxstylestrong{you can decide on which one of the two operating systems you want to use}. Unlike the case described above in which Linux runs \sphinxstylestrong{within} Windows, in this case to switch from one operating system to the other you will always have to reboot the machine.

\sphinxAtStartPar
The installation of a dual boot system is easy, in principle, but there are a few things that you have to be careful on, like creating a partition of the hard disk on which you want to install Linux. If you make a mistake here you might end up losing Windows for example. My advice is to read carefully one of the following (or other guides) before attempting this:
\begin{itemize}
\item {} 
\sphinxAtStartPar
\sphinxhref{https://itsfoss.com/install-ubuntu-dual-boot-mode-windows/}{How To Install Ubuntu Along With Windows}%
\begin{footnote}[20]\sphinxAtStartFootnote
\sphinxnolinkurl{https://itsfoss.com/install-ubuntu-dual-boot-mode-windows/}
%
\end{footnote}

\item {} 
\sphinxAtStartPar
\sphinxhref{https://www.youtube.com/watch?v=-iSAyiicyQY}{How to Dual Boot Ubuntu 20.04 LTS and Windows 10}%
\begin{footnote}[21]\sphinxAtStartFootnote
\sphinxnolinkurl{https://www.youtube.com/watch?v=-iSAyiicyQY}
%
\end{footnote}

\item {} 
\sphinxAtStartPar
\sphinxhref{https://averagelinuxuser.com/dualboot-linux-windows/}{How to Dual boot Windows 10 and Linux (Beginner’s Guide)}%
\begin{footnote}[22]\sphinxAtStartFootnote
\sphinxnolinkurl{https://averagelinuxuser.com/dualboot-linux-windows/}
%
\end{footnote}

\end{itemize}

\begin{sphinxuseclass}{nbinput}
\begin{sphinxuseclass}{nblast}
{
\begin{sphinxVerbatim}[commandchars=\\\{\}]
\llap{\color{nbsphinxin}[ ]:\,\hspace{\fboxrule}\hspace{\fboxsep}}
\end{sphinxVerbatim}
}

\end{sphinxuseclass}
\end{sphinxuseclass}



\chapter{ }
\label{\detokenize{toc:space}}


\renewcommand{\indexname}{Index}
\printindex
\end{document}